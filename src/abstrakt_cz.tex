%!TEX root = ../main.tex

\begin{changemargin}{0.8cm}{0.8cm}

~\vfill{}

\section*{Abstrakt}
\vskip 0.5em

\sloppy
Tato práce je zaměřená na problematiku sledování lesní cesty pomocí obrázku z monokulární kamery, připevněné na bezpilotní helikoptéře nebo pozemním vozidle. Je představen systém, řešící úlohu navigace podél stezky v lese. Toho bylo dosaženo s využitím klasifikační hluboké konvoluční neuronové sítě pro určení směru natočení helikoptéry vzhledem k cestě. Systém byl implementován s minimálním zpožděním, aby mohl být zapojen ve zpětné vazbě s plánováním trajektorie helikoptéry v rámci MRS UAV systému. Součastí implementace je algoritmus na generování bodů trajektorie pro plánovač. Výkon a robustnost byly otestovány v simulaci a následně během experimentů v reálném světe. Implementovaný systém prokázal dobré praktické výsledky a může být použit jako výchozí bod pro komplexnější navigační a průzkumné aplikace.

\vskip 1em

{\bf Klíčová slova} \KlicovaSlova

\vskip 2.5cm

\end{changemargin}
